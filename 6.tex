\textbf{Опр 1} Случайная величина - это функция $\xi : \Omega \to \mathbb{R} $
если $\mathcal{F} = 2^\Omega$ и  $\Omega$ или конечное или счетное. {\color{red} $P(A)
= \sum\limits_{w\in A}^{} P(w)$ Это не требовал Иван Генрихович в лекциях прошлого года, не понятно зачем это вообще требовать}

\textbf{Опр 1*} Случайная величина - это функция $\xi : \Omega \to \mathbb{R} $
такая что $\forall a \in \mathbb{R} : \xi^{-1}((-\infty, a]) \in \mathcal{F}$
(Измеримость функции)

\textbf{Опр 2} Математическое ожидание от случайной величины - это $E\xi =
\sum\limits_{w\in \Omega}^{} \xi(w)P(w)$ \\
Если счетное количество $E\xi = \sum\limits_{x\in \xi(\Omega)}^{} P(\xi = x)$

\textbf{Опр 2*} $E\xi = \bigintss\limits_{x \in \Omega}^{} \xi(w) d P$ 

\textbf{Опр 3} Распределение случайной величины - это значение сл. в. и
вероятности $P(\xi = x_i) = p_i$

\textbf{Опр 3*} Распределение случайной величины($P_\xi$) - это вероятностная мера на $\Omega$.  Такое что $P_\xi(B) = P(\xi \in B)$

\textbf{Свойства матожидания} $\xi \geqslant 0 \Rightarrow E\xi \geqslant 0$. Линейность. $\xi \geqslant \eta \Rightarrow E(\xi)  \geqslant E(\eta)$. $|E\xi| <= E|\xi|$. Неравенство Коши-Бунековского  $(E\xi\eta)^2  \leqslant E\xi^2E\eta^2 $ равенство $\Leftrightarrow \exists a,b\in\mathbb{R}: a\xi + b\eta = 0$

Доказательство $\forall t\in\mathbb{R} $
