Математическая модель в которой мы будем работать - это $(\Omega, \mathcal{F},
P)$, где $\Omega$ - это пространство элементарных исходов, $\mathcal{F}$ - это
$\sigma$-алгебра событий, $P$ - это $\sigma$-аддитивная вероятностная мера. 

В этом курсе предмет нашего исследования - это случайный эксперемент.
\begin{enumerate}
  \item повторяемость
  \item отсутсвие детерминистической регулярности
  \item статистическая устойчивость частот
\end{enumerate}

\textbf{ Элементарный исход} - это результат случайного эксперемента.($\Omega$)

\textbf{ События} - это множество элементарных исходов(нам обычно интересно именно множество,
например множество оценок на сессии, чтобы не упал средний
балл)($\mathcal{F}$)

\textbf{ Вероятность} - это частота события, если мы будем много
раз повторять случайный эксперемент. Идеализация!

Не стоит воспринимать $P(A) = \lim\limits_{n\to\infty } \cfrac{N(A)}{n} $ потому что это не про жизн!

