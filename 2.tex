\section{}

\textbf{Опр} будем называть модели дискретными, если $\#\Omega$ не более чем счетно.

\textbf{Пример} будем рассматривать модель где у нас есть мешок, в котором есть шарики ($M$ - белых и $N - M$ - черных)

Эксперементы могут быть с(без) возвращением(я) и с(без) учетом(а) порядка.

Тогда элементарный исход это или кортеж или множество. А возвращение влияет на
мощность элементарных исходов. Комбу надо было учить!

\textbf{Классическая теория вероятности} занимается дискретными моделями в
которых элементарные исходы равновероятны. \textbf{\textit{Замечание} }классическая модель занимается только конечными моделями, так как сумма вероятностей должна равнятся $1$.

