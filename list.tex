\documentclass[a4paper, 12pt]{article}
\usepackage[utf8]{inputenc}
\pagestyle{empty}
\usepackage[usenames]{color} %used for font color
\usepackage[russian]{babel}
\usepackage{amssymb} %maths
\usepackage{amsmath} %maths
\usepackage{wrapfig}
\usepackage{xcolor}

\usepackage{chapterbib}

\usepackage{natbib}
\usepackage{graphicx}
\usepackage{hyperref}
\usepackage{bigints}

\usepackage[normalem]{ulem}  % для зачекивания текста

\definecolor{linkcolor}{HTML}{0000FF} % цвет ссылок
\definecolor{urlcolor}{HTML}{0000FF} % цвет гиперссылок
\hypersetup{pdfstartview=FitH,  linkcolor=linkcolor,urlcolor=urlcolor, colorlinks=true}

\oddsidemargin = 0 pt
\textwidth = 18 cm
\marginparsep = 0 pt
\marginparwidth = 0 pt
\hoffset = -0.41 in

\headheight = 0 pt
\headsep = 0 pt
\topmargin = 0 pt
\voffset = -0.4 in
\textheight = 10.511 in

\graphicspath{{pictures/}}
\DeclareGraphicsExtensions{.pdf,.png,.jpg}

\begin{document}
  \begin{enumerate}
    \item В классической модели конечное количество элементарных исходов
    \item Независимость событий
    \item Условная вероятность. Формула Байоса.
    \item Независимость в совокупности
    \item Попарная независимость
    \item Случайная величина(настоящее определение)
    \item Схема Бернулли
    \item Независимость случайных величин
    \item Распределение
    \item Распределение Бенулли, биномиальное, геометрическое, Пуассона(у нас еще была теорема без доказательства о приближении Пуассоновского и биномиального)
    \item Математическое ожидание(+ формула для подсчета) Свойства: Если сл. в. $\geqslant 0$, то мат. ож. $\geqslant 0$, линейность, $\xi \geqslant  \eta \Rightarrow \mathbb{E} \xi > \mathbb{E} \eta$, $|\mathbb{E} \xi| \leqslant \mathbb{E} |\xi|$. {Неравенство Коши-Бунековского.} Если сл. в. не зависимы, то матожидание произведения равно произведению матожиданий
    \item Дисперсия. Свойства: положительность. Пофиг на + константу. Выносим константу с квадратом
    \item Ковариация.
    \item Неравенство Маркова, неравенство Чебышева, закон больших чисел. Центральная предельная(б/д)


  \end{enumerate}
\end{document}
